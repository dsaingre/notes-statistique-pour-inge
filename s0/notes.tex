\documentclass[10pt]{scrartcl}

%----------------
% CONFIG
%----------------

\usepackage[utf8]{inputenc}
\usepackage[french]{babel}
\usepackage[T1]{fontenc}
\usepackage[margin=0pt, landscape]{geometry}
\usepackage{graphicx}
\usepackage{color}
\usepackage{amsfonts} % for Number set
\usepackage{amsmath}

\usepackage[
colorlink=false,
pdftitle={Resume S1},
pdfauthor={Dimitri Saingre},
pdfsubject={Statistique Descriptive}
]{hyperref}

\setlength{\unitlength}{1mm}
\setlength{\parindent}{0pt}

\newcommand{\sectiontitle}[1]{\paragraph{#1} \ \\}

%----------------

\begin{document}

\begin{picture}(297,210)

%----------------
% TITLE
%----------------

\put(10,200){
\begin{minipage}[t]{85mm}
\section*{Thème 0 : Statistique Descriptive}
\end{minipage}
}

%----------------
% FIRST COLUMN
%----------------

\put(10,180){
\begin{minipage}[t]{85mm}

\sectiontitle{Introduction}
\textbf{Statistique descriptive} : permet de décrire les données à l'aide de graphiques et de paramètres d'une façon compréhensible et utilisable\\
\textbf{Probabilité} : permet de modéliser efficacement les phénomènes étudiés en statistiques \\
\textbf{Statistique inférencielle} : permet de faire des prévisions ou généralisations à toute une population à partir d'échantillons\\
\textbf{Régression linéaire} : permet d'étudier la relation existante entre deux variables. Met en place des modèles de prévisions et des outils pour
valider ceux ci\\

\sectiontitle{Vocabulaire}
\begin{tabular}{|l|c|r|}
  \hline
  \textbf{Ensembliste} & \textbf{Statistique} \\
  \hline
  Ensemble & Population ($\Omega$) \\
  Application & Variable / Caractère \\
  Elément & Individu / unité statistique \\
  Sous-Element & Sous-population \\
  Cardinal & Effectif \\
  \hline
\end{tabular}
\textbf{Fréquence} d'une sous population $E$ de $\Omega$ : 
$f(E) = \frac{Card(E)}{Card(\Omega)} \in [0,1]$ \\

\sectiontitle{Variables ou caractères}
Variables \textbf{qualitatives} : appartenance à une catégorie\\
Variables \textbf{quantitative}/\textbf{numériques} : taille, poids, volume...\\
Variables \textbf{discrète} : nombre fini ou indéfini dénombrable de valeurs observées\\
Soit une variable discrète $X$, l'ensemble des valeurs (modalités) prises par $X$ est l'ensemble :\\
$X(\Omega) = \{x_{1}, x_{2}, \dots, x_{n} \dots\} = \{x_{i}, i \in \mathbb{N}\}$ \\

\sectiontitle{Loi d'une variable quantitative, fonction de répartition}
La \textbf{loi} ou \textbf{distribution empirique} d'une variable $X$ sur $\Omega$ est la 
donnée de la fréquence de chaque classe définie par la variable $X$
\begin{itemize}
\item Si $X$ quantitative ou qualitative discrète, sa loi est définie par la fréquence de
chaque sous-population du type $\{X = x_{i}\} = \{\omega \in \Omega, X(\omega) = x_{i}\}$ ;
\end{itemize}

\end{minipage}
}

%----------------
% SECOND COLUMN
%----------------

\put(105,180){
\begin{minipage}[t]{85mm}
\begin{itemize}
\item Si $X$ continue et si les valeurs possibles de $X$ sont réparties en classes $C_{i}$,
la loi est la donnée de chaque fréquence des sous-populations $\{X \in C_{i}\} = \{\omega \in \Omega, X(\omega) \in C_{i}\}$ 
\end{itemize}
La \textbf{fonction de répartition empirique} de $X$ est la fonction, notée $F_{x}$, qui à $x \in \mathbb{R}$ associe
la fréquence de la sous-population $\{X \leq x\}$ :\\
\begin{equation*}
  \begin{split}
    F_{X} : & \mathbb{R} \to \mathbb{R} \\
    & x \mapsto F_{X}(x) = \frac{Card\{ \omega \in \Omega, X(\omega) \leq x\}}{Card \Omega}
  \end{split}
\end{equation*} 

\sectiontitle{Grandeurs statistiques usuelles}
La \textbf{moyenne} du caractère $X$ est la quantité \\
$\bar{x} = \frac{1}{n} \sum\limits_{i=1}^{n} x_{i}$\\
La moyenne est une statistique \textit{peu robuste} (sensible aux valeurs extrêmes)\\
\textbf{Proposition : } si $Y = aX + b$ avec $a,b \in \mathbb{R}$, alors : $\bar{y} = a\bar{x} + b$\\
Le \textbf{mode} ou \textbf{classe modale} d'une distribution statistique est la valeur ou la classe
du caractère qui correspond à la plus grande fréquence.\\
La \textbf{mediane} du caractère $X$ est la valeur $M_{e}$ telle que, en notant $f(\dots)$ la fréquence :
$f(\{X \leq M_{e}\}) \geq \frac{1}{2} et f(\{X \geq M_{e}\}) \geq \frac{1}{2}$\\
Les \textbf{quartiles} $Q_{1}, Q_{2} et Q_{3}$ sont les valeurs permettant de diviser la population en quatre sous-populations
d'effectif égaux, représentant chacune 25\% de la population totale.\\
L'\textbf{étendue} est la différence entre les valeurs extrêmes du caractère : $\omega = x_{max} - x_{min}$\\
La \textbf{variance} de la variable $X$ est la quantitié $\sigma^2 = \frac{1}{n} \sum\limits_{i=1}^{n} (x_{i} - \bar{x})^{2} = \frac{1}{n} \sum\limits_{i=1}^{n} x_{i}^{2} - \bar{x}^{2}$ 
représentant la moyenne des carrés des écarts entre les observations et leur moyenne\\
\textbf{Proposition: } transformation affine sur la variance : si on pose $Y = aX+b$, $\sigma_{Y}^2 = a^2\sigma_{X}^2$\\
L'\textbf{écart-type} de $X$ est la racine carrée $\sigma$ de la variance

\end{minipage}
}

%----------------
% THIRD COLUMN
%----------------

\put(200,180){
\begin{minipage}[t]{85mm}
\sectiontitle{Distributions à deux caractères}
L'\textbf{effectif marginal en} $X$ et la \textbf{fréquence marginale en} $X$ de la classe $C_{i}$ :\\
$n_{i.} = \sum\limits_{j=1}^{s} n_{ij}$ et $f_{i.} = \frac{n_{i.}}{n} = \sum\limits_{j=1}^{s} f_{ij}$\\
La \textbf{loi conditionnelle de} $Y$ \textbf{sachant} $X \in C_{i}$ est la donnée, pour tout $j\in\{1,\dots,s\}$
des fréquences relatives des classes $D_{j}$ par rapport à $C_{i}$ : $f_{j/i} = \frac{n_{ij}}{n_{i.}} = \frac{f_{ij}}{f_{i.}}$\\
Les deux variables $X$ et $Y$ sont dites \textbf{indépendantes} si la loi conditionnelle de $Y$ sachant $X \in C_{i}$ ne 
dépend pas de $i$\\

\sectiontitle{Cas de deux variances quantitatives}
La \textbf{covariance} de deux variables quantitatives $X$ et $Y$ est : $\mathbb{C}$ov$(X,Y) = \frac{1}{n}\sum\limits_{i=1}^{n}(x_{i} - \bar{x})(y_{i} - \bar{y})$\\
La covariance permet de quantifier la liaison entre les deux variables (positive = même sens = \textbf{liaison positive}, 
négatif = sens contraires = \textbf{liaison négative})\\
\textbf{Propriétés de la covariance: } \\
\textit{(1) La covariance est symétrique :} $\mathbb{C}$ov $(X,Y) = \mathbb{C}$ov $(Y,X)$\\
\textit{(2) Covariance de X avec elle-même :} $\mathbb{C}$ov $(X,X) = \mathbb{V}(X)$\\
\textit{(3) Transformation affine :} $\mathbb{C}$ov $(aX+b,cY+d) = ac\mathbb{C}$ov $(Y,X)$\\
\textit{(4) Variance d'une somme :} $\mathbb{V}(X+Y) = \mathbb{V}(X) + 2\mathbb{C}$ov$(X,Y) + \mathbb{V}(Y)$\\
\textit{(5) Inégalité de Cauchy-Schwartz :} $\lvert \mathbb{C}$ov $(X,Y)\rvert \leq \sigma(X) \sigma(Y)$ avec égalité
si et seulement si il existe une relation affine entre $X$ et $Y : Y = aX+b$ ou $X=cY+d$ \\
\textit{(6) Cas de variables indépendantes :} si $X$ et $Y$ sont indépendantes, leur covariance est nulle. La réciproque est \underline{fausse} \\
Lorsque deux variables ont une covariance nulle, on dit qu'elles sont \textbf{décorrélées}.

\end{minipage}
}

\end{picture}

\end{document}
