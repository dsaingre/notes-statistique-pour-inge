\documentclass[10pt]{scrartcl}

%----------------
% CONFIG
%----------------

\usepackage[utf8]{inputenc}
\usepackage[french]{babel}
\usepackage[T1]{fontenc}
\usepackage[margin=0pt, landscape]{geometry}
\usepackage{graphicx}
\usepackage{color}
\usepackage{amsfonts} % for Number set
\usepackage{amsmath}

\usepackage[
colorlink=false,
pdftitle={Resume S1},
pdfauthor={Dimitri Saingre},
pdfsubject={Notions de probabilités}
]{hyperref}

\setlength{\unitlength}{1mm}
\setlength{\parindent}{0pt}

\newcommand{\sectiontitle}[1]{\paragraph{#1} \ \\}

%----------------

\begin{document}

\begin{picture}(297,210)

%----------------
% TITLE
%----------------

\put(10,200){
\begin{minipage}[t]{85mm}
\section*{Thème 1 : Notions de probabilités}
\end{minipage}
}

%----------------
% FIRST COLUMN
%----------------

\put(10,180){
\begin{minipage}[t]{85mm}
\sectiontitle{Expériences aléatoires et modèles probabilistes}
\textbf{Axiomatique de Kolmogorov} : l'ensemble $\Omega$ de tous les résultats possibles d'une expérience 
spécifiée par un protocole expérimental donné est appelé \textbf{univers}. On dira aussi que $\Omega$ est
l'espace des états ou espaces des possibles de l'expérience aléatoire. \\
\textbf{Espace probabilisable} : un espace probabilisable est un couple $(\Omega,T)$, où $\Omega$ est un 
ensemble et $T$ une tribu de $\Omega$, càd un ensemble de parties de $\Omega$ vérifiant les propriétés
suivantes : \\
\begin{itemize}
\item $\Omega \in T$
\item Si $A \in T$, alors $\bar{A} \in T$ où $\bar{A}$ est le complémentaire de $A$ dans $\Omega$
\item Si $(A_n)_{n\in\mathbb{N}^*}$ est une suite d'éléments de $T$, alors $\bigcup\limits_{n=1}^{\infty}A_n \in T$
\end{itemize}
Les éléments de $T$ sont appelés événements. En particulier, pour tout $\omega \in \Omega$, le singleton
$\{\omega\}$ est appelé événement élémentaire.\\
Un \textbf{espace probabilisé} est un triplet $(\Omega,T,\mathbb{P})$ où $(\Omega,T)$ est un espace probabilisable
et $\mathbb{P}$ une mesure de probabilité sur $T$, càd une application de $T$ dans $[0,1]$, telle que :
\begin{itemize}
\item $\mathbb{P}(\Omega) = 1$ \textit{[Condition de normalisation]}
\item Soit $(A_n)_{n\in\mathbb{N}^*}$ une suite d'événements disjoints 2 à 2, 
$\mathbb{P}(\bigcup\limits_{n=1}^\infty A_n) = \sum\limits_{n=1}^\infty \mathbb{P}(A_n)$ \textit{[$\sigma$-additivité]}
\end{itemize}

\end{minipage}
}

%----------------
% SECOND COLUMN
%----------------

\put(105,180){
\begin{minipage}[t]{85mm}

\end{minipage}
}

%----------------
% THIRD COLUMN
%----------------

\put(200,180){
\begin{minipage}[t]{85mm}

\end{minipage}
}

\end{picture}

\end{document}
